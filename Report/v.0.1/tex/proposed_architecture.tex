\section{Design Goals}

%Overview presents a bird?s-eye view of the software architecture and briefly describes the assignment of functionality to each subsystem.
\section{Overview}

% Subsystem decomposition describes the decomposition into subsystems and the responsibilities of each. This is the main product of system design.
\section{Subsystem Decomposition}
\label{sec:subsystems}


%Persistent data management describes the persistent data stored by the system and the data management infrastructure required for it. This section typically includes the description of data schemes, the selection of a database, and the description of the encapsulation of the database.
\section{Persistent Data Management}

% Access control and security describes the user model of the system in terms of an access matrix. This section also describes security issues, such as the selection of an authentication mechanism, the use of encryption, and the management of keys.
\section{Access Control and Security}

%Global software control describes how the global software control is implemented. In particular, this section should describe how requests are initiated and how subsystems synchronize. This section should list and address synchronization and concurrency issues.
\section{Global Software Control}

%Boundary conditions describes the start-up, shutdown, and error behavior of the system. (If new use cases are discovered for system administration, these should be included in the requirements analysis document, not in this section.)
\section{Boundary Conditions}


