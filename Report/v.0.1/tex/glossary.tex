
% ACRONYM ENTRIES
% === Acronyms ===
%\usepackage{acronym}
\newacro{IT}{Information Technology} 


\newacronym{tls}{TLS}{\gls{transport_layer_security}}

\newacronym{mg}{mg}{\gls{milligram}}

% GLOSSARY ENTRIES

\newglossaryentry{patient_dashboard}
{
    name = Patient Dashboard,
    description = {The primary screen of the phone application, which provides an overview of the patient's data and allow access to all the features and sub-screens of the application.}
}

\newglossaryentry{self_assessment}
{
    name = Self-assessment,
    description = {The daily assessment done through filling out a questionnaire regarding how the patients are doing and feeling.}
}


%\newglossaryentry{naiive}  % \gls{naiive}
%{
%  name=na\"ve,
%  description={ASDIs a French loanword (adjective, form of naïf)
%               indicating having or showing a lack of experience,
%               understanding or sophistication},
%  sort=naive % Value to sort by
%}

%\newglossaryentry{Linux}  % \gls{Linux}
%{
%  name=Linux,
%  description={is a generic term referring to the family of Unix-like
%               computer operating systems that use the Linux kernel},
%  plural=Linuces, % \glspl{Linux}
%  user1={text1}, % \glsuseri{Linux}
%  user2={text2}, % \glsuserii{Linux}
%  symbol={\ensuremath{\Omega} % \glssumbol{Linux}
%}

%\newacronym{acr1}{VCS}{Version control system.} % \gls{acr1}

%\newacronym{acr2}{TXT}{Maybe some file, who knows?}  % \gls{acr2}

%\gls{naiive} = na\"ve

%\gls{Linux} % Linux
%\glspl{Linux} % Linuces
%\glsuseri{Linux} % text1
%\glsuserii{Linux} % text2
%\glssumbol{Linux} % \ensuremath{\Omega}

%\gls{acr1} % VCS
%\glspl{acr1} % VCSs
%\acrlong{acr1} % Version control system.

%\gls{acr2} % TXT
%\glspl{acr2} % TXTs
%\acrlong{acr2} % Maybe some file, who knows?

% To emphasize a number of a specific page use the format option.
%\gls[format=hyperbf]{acr1}